\documentclass[11pt,a4paper,bibliography=totoc,listof=totoc,pointlessnumbers,open=any]{scrbook}

\usepackage[left=3.0cm,right=2.5cm,top=2.5cm,bottom=3cm]{geometry}	% Seitenränder
\headsep = 1.0cm
\footskip = 1.0cm
                          
\addtokomafont{disposition}{\rmfamily}
\addtokomafont{chapter}{\rmfamily}
\addtokomafont{section}{\rmfamily}
\addtokomafont{subsection}{\rmfamily}
\addtokomafont{subsubsection}{\rmfamily}

\usepackage{setspace} 						% Zeilenabstand 1,5
\onehalfspacing

\usepackage[utf8]{inputenc}					
\usepackage[english]{babel}					% Silbentrennung
\usepackage{amsmath}						% mathematische Formeln
\usepackage{amsfonts}						% mathematische Schriften
\usepackage{amssymb}						% mathematische Symbole
\usepackage{graphicx}						% Einbinden von Graphiken
\usepackage[version=4]{mhchem}
\usepackage{subfigure}						% Ermöglicht ein Subfigure
\usepackage{pdfpages} 						
\usepackage{acronym}						% Symbol- und Abkürzungsverzeichnis
\usepackage[figuresright]{rotating}
\usepackage{subcaption}

%\usepackage[table]{xcolor}
%\usepackage[T1]{fontenc}
%\usepackage{mathptmx}
%\usepackage{colortbl}
%-------MATLAB-------------------------------------------------

%\usepackage{matlab-prettifier}
%\definecolor{number}{gray}{0.6}
%\definecolor{back}{gray}{0.95}

\usepackage{color}                  %\definecolor
\usepackage{listings}               %Programmcodeumgebung für Matlab Code

\usepackage{units} 							% Schreiben von Zahlen mit Einheit mit kleinerem Abstand
\usepackage{fix-cm}							% Schriften größer als \Huge
\usepackage{multirow} 						% Tabelle - Spalten verbinden
\usepackage{array}
\usepackage{here}							% Gleitobjekte "fest setzen"

\usepackage{float}							% ermöglicht Gleitobjekte
\usepackage[font=footnotesize,labelfont=bf]{caption}						% ermöglicht Umbrüche in Bildunterschriften

\usepackage{chngcntr}						% Nummerierung nach Kapitel für Abbildungen, Tabellen, Formeln
\counterwithin{figure}{chapter}
\counterwithin{table}{chapter}
\counterwithin{equation}{chapter}

\usepackage{longtable}						% lange Tabellen über mehrere Seiten
\usepackage{multirow}
\usepackage{booktabs}

\usepackage[headsepline]{scrlayer-scrpage}
\pagestyle{scrheadings}
\automark[section]{chapter}


\usepackage{longtable}						% lange Tabellen über mehrere Seiten

\usepackage[tracking=true]{microtype}		% Anpassen der Abstände zwischen den Buchstaben für Blocksatz

\graphicspath{{Figures/}} 					% Pfad für Bidler
\def\figurename{Figure } 					% keyword for the captions

\DeclareMicrotypeSet*[tracking]{my} %
{ font = */*/*/sc/* }% 
\SetTracking{ encoding = *, shape = sc }{ 45 }% Hier wird festgelegt, dass alle Passagen in Kapitälchen automatisch leicht gesperrt werden. 

\usepackage[colorlinks=false, pdfborder={0 0 0}]{hyperref} % Verlinkung von Inhalts- und Abbildungsverzeichnis 
\usepackage[all]{hypcap} 

\usepackage[numbers]{natbib}				% Paket zur Einbindung des Literaturverzeichnisses 

\usepackage{scrhack}						% Behebt Probleme zwischen den einzelnen Paketen

\begin{document}
\pagestyle{scrheadings}

\newpage
\thispagestyle{empty}
\begin{flushleft}
	\includegraphics[width=0.5\textwidth]{Titel/OTH_Logo_RAI.jpg}
\end{flushleft}

\begin{centering}
	\bigskip 
	\bigskip
	\bigskip
	\bigskip		
	\bigskip
	\huge\textrm{\textbf{Modelling the mechanical behaviour of the thoracic aorta for a stent-aorta interaction in Abaqus }}\\
	\bigskip
	\bigskip
	\bigskip 
	\bigskip 
	\bigskip
		\bigskip
		\bigskip 
		\bigskip
		\bigskip
	\LARGE\textrm{\textbf{ABSCHLUSSARBEIT}}\\
	\bigskip
	\small\textrm{zur Erlangung des akademischen Grads}\\
	\smallskip
	\large\textrm{\textbf{\glqq Bachelor of Science (B.Sc.)\grqq}}\\
	\smallskip
	\small\textrm{an der}\\
	\smallskip
	\large\textrm{\textbf{Ostbayerischen Technischen Hochschule Regensburg}}\\
	\smallskip
	\small\textrm{Fakultät Maschinenbau}\\
	\smallskip
	\small\textrm{Computational Mechanics and Materials Lab}\\
	\smallskip
	\small\textrm{im Studiengang}\\
	\smallskip
	\large\textrm{\textbf{Bachelor Biomedical Engineering}}
	\bigskip 
	\bigskip 
	\bigskip
	\bigskip
	\bigskip 
	\bigskip
	\bigskip 
	\bigskip
	\bigskip 
    \begin{table}[H]
    	\flushleft
    	\renewcommand{\arraystretch}{1.2}
    	\begin{tabular}{ll}
    	\textbf{Autor:} & \textrm{Nova Sassin} \\ 
    	 & \textrm{Matr.-Nr: } 3215867 \\
    	\textbf{Betreuer:} & \textrm{Philipp Marx, M.Sc.} \\ 
    	\textbf{Aufgabenstellerin:} & \textrm{Prof. Dr.-Ing. Aida Nonn}
    	\end{tabular}
    \end{table}
\end{centering}

\newpage
\tableofcontents 



\end{document}
